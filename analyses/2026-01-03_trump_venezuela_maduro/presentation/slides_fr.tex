% =============================================================================
% Conférence de presse Trump sur le Venezuela
% Analyse du discours politique
% Auteur: Antoine Lemor
% Date: Janvier 2026
% =============================================================================

\documentclass[aspectratio=169,12pt]{beamer}

\usetheme{default}
\usecolortheme{default}

% Remove navigation symbols
\setbeamertemplate{navigation symbols}{}

% Colors - refined palette
\definecolor{bglight}{RGB}{250, 250, 252}
\definecolor{primary}{RGB}{30, 41, 59}
\definecolor{accent}{RGB}{59, 130, 246}
\definecolor{danger}{RGB}{239, 68, 68}
\definecolor{success}{RGB}{34, 197, 94}
\definecolor{muted}{RGB}{148, 163, 184}
\definecolor{subtle}{RGB}{226, 232, 240}

% Beamer colors
\setbeamercolor{background canvas}{bg=bglight}
\setbeamercolor{frametitle}{fg=primary}
\setbeamercolor{title}{fg=primary}
\setbeamercolor{subtitle}{fg=muted}
\setbeamercolor{author}{fg=primary}
\setbeamercolor{date}{fg=muted}
\setbeamercolor{section in toc}{fg=primary}
\setbeamercolor{itemize item}{fg=accent}
\setbeamercolor{itemize subitem}{fg=accent}
\setbeamercolor{enumerate item}{fg=accent}

% Fonts
\setbeamerfont{title}{size=\LARGE, series=\bfseries}
\setbeamerfont{subtitle}{size=\normalsize}
\setbeamerfont{frametitle}{size=\large, series=\bfseries}
\setbeamerfont{framesubtitle}{size=\small}

% Frame title format
\setbeamertemplate{frametitle}{
    \vspace{0.8em}
    \insertframetitle
    \par
    \usebeamerfont{framesubtitle}\usebeamercolor[fg]{framesubtitle}\insertframesubtitle
    \vspace{0.3em}
    \textcolor{subtle}{\rule{\textwidth}{0.5pt}}
    \vspace{0.3em}
}

% Packages
\usepackage[utf8]{inputenc}
\usepackage[T1]{fontenc}
\usepackage[french]{babel}
\usepackage{graphicx}
\usepackage{booktabs}
\usepackage{tikz}
\usepackage{xcolor}
\usepackage{fontawesome5}
\usepackage{hyperref}
\hypersetup{colorlinks=true,linkcolor=primary,urlcolor=accent}

\usetikzlibrary{calc,shapes,positioning,shadows}

\graphicspath{{../output/figures/}}

% Custom commands
\newcommand{\emphdata}[1]{\textbf{\textcolor{accent}{#1}}}
\newcommand{\emphwarn}[1]{\textbf{\textcolor{danger}{#1}}}
\newcommand{\emphok}[1]{\textbf{\textcolor{success}{#1}}}
\newcommand{\muted}[1]{\textcolor{muted}{#1}}
\newcommand{\figframe}[1]{%
    \begin{tikzpicture}
        \node[inner sep=0pt] (fig) {#1};
        \draw[subtle, line width=0.5pt, rounded corners=2pt]
            ($(fig.south west)+(-2pt,-2pt)$) rectangle ($(fig.north east)+(2pt,2pt)$);
    \end{tikzpicture}%
}

% Analysis counter
\newcounter{analysisnum}
\setcounter{analysisnum}{0}

% For conditional check
\usepackage{xstring}

% Section page
\AtBeginSection[]{
  \IfStrEq{\insertsectionhead}{Conclusion}{%
    % Conclusion section - no analysis number
    \begin{frame}
    \vfill
    \centering
    \begin{tikzpicture}
      \node[fill=white, rounded corners=8pt, inner sep=20pt, drop shadow={shadow xshift=1pt, shadow yshift=-1pt, opacity=0.1}] {
        \begin{minipage}{0.6\textwidth}
          \centering
          {\LARGE\bfseries\textcolor{primary}{\insertsectionhead}}
        \end{minipage}
      };
    \end{tikzpicture}
    \vfill
    \end{frame}
  }{%
    % Regular analysis section
    \stepcounter{analysisnum}
    \begin{frame}
    \vfill
    \centering
    \begin{tikzpicture}
      \node[fill=white, rounded corners=8pt, inner sep=20pt, drop shadow={shadow xshift=1pt, shadow yshift=-1pt, opacity=0.1}] {
        \begin{minipage}{0.6\textwidth}
          \centering
          \textcolor{muted}{\small ANALYSE N°\theanalysisnum}\\[0.5em]
          {\LARGE\bfseries\textcolor{primary}{\insertsectionhead}}
        \end{minipage}
      };
    \end{tikzpicture}
    \vfill
    \end{frame}
  }
}

% Metadata
\title{Trump et le Venezuela}
\subtitle{Ce que révèle l'analyse de la conférence de presse du 3 janvier 2026}

\date{3 janvier 2026}
\institute{}

\begin{document}

% =============================================================================
% TITLE
% =============================================================================
\begin{frame}[plain]
    \vfill
    \begin{center}
        \begin{tikzpicture}
            \node[fill=white, rounded corners=12pt, inner sep=30pt, drop shadow={shadow xshift=2pt, shadow yshift=-2pt, opacity=0.15}] {
                \begin{minipage}{0.75\textwidth}
                    \centering
                    {\small\textcolor{accent}{\faChartBar\ NLP \& science politique}}\\[1em]
                    {\LARGE\bfseries\textcolor{primary}{Trump et le Venezuela}}\\[0.8em]
                    {\normalsize\textcolor{muted}{Ce que révèle l'analyse de la \\conférence de presse du 3 janvier 2026}}\\[1.5em]
                    \textcolor{subtle}{\rule{0.3\textwidth}{0.5pt}}\\[1.2em]
                    {\footnotesize\textcolor{muted}{5 janvier 2026}}
                \end{minipage}
            };
        \end{tikzpicture}
    \end{center}
    \vfill
\end{frame}

% =============================================================================
% INTRODUCTION
% =============================================================================
\begin{frame}[c]{Une conférence de presse, quatre personnes}
    \centering
    \begin{columns}[c]
        \begin{column}{0.60\textwidth}
            \begin{tikzpicture}
                \node[fill=white, rounded corners=6pt, inner sep=15pt, drop shadow={opacity=0.08}] {
                    \begin{minipage}{0.85\textwidth}
                        {\large\bfseries 3 janvier 2026}\\[0.8em]
                        Conférence de presse à la Maison Blanche suite à l'opération militaire au Venezuela.\\[1.5em]
                        \muted{\footnotesize \faDatabase\ 858 phrases annotées}\\
                        \muted{\footnotesize \faNewspaper\ Transcription textuelle d'Associated Press}\\
                        \muted{\footnotesize \faYoutube\ \href{https://www.youtube.com/watch?v=ezYNnFETXk0}{Vidéo source}}
                    \end{minipage}
                };
            \end{tikzpicture}
        \end{column}
        \begin{column}{0.40\textwidth}
            \begin{tikzpicture}[every node/.style={inner sep=8pt}]
                \node[fill=white, rounded corners=4pt] at (0,0) {
                    \begin{tabular}{cl}
                        \textcolor{red!80!black}{\faSquare} & \textbf{Donald Trump} \\[-0.2em]
                        & \muted{\scriptsize Président} \\[0.6em]
                        \textcolor{blue!70!black}{\faSquare} & \textbf{Marco Rubio} \\[-0.2em]
                        & \muted{\scriptsize Secrétaire d'État} \\[0.6em]
                        \textcolor{green!60!black}{\faSquare} & \textbf{Pete Hegseth} \\[-0.2em]
                        & \muted{\scriptsize Secrétaire à la Défense} \\[0.6em]
                        \textcolor{purple!70!black}{\faSquare} & \textbf{Dan Caine} \\[-0.2em]
                        & \muted{\scriptsize Chef d'état-major} \\
                    \end{tabular}
                };
            \end{tikzpicture}
        \end{column}
    \end{columns}
\end{frame}

% -----------------------------------------------------------------------------
\begin{frame}{Méthodologie}
    \vspace{0.5em}

    \centering
    \begin{tikzpicture}[
        node distance=1.2cm,
        box/.style={rectangle, rounded corners=6pt, minimum width=2.6cm, minimum height=1.2cm, align=center, font=\scriptsize},
        arrow/.style={->, >=stealth, thick, muted}
    ]
        % Boxes
        \node[box, fill=accent!15] (video) {\faYoutube\ \textbf{Vidéo}\\[0.1em]\muted{\tiny AP}};
        \node[box, fill=success!15, right=of video] (transcription) {\faFile\ \textbf{Transcription}\\[0.1em]\muted{\tiny 858 phrases}};
        \node[box, fill=danger!15, right=of transcription] (annotation) {\faTags\ \textbf{Annotation}\\[0.1em]\muted{\tiny LLM}};
        \node[box, fill=purple!15, right=of annotation] (analyse) {\faChartBar\ \textbf{Analyse}};

        % Arrows
        \draw[arrow] (video) -- (transcription);
        \draw[arrow] (transcription) -- (annotation);
        \draw[arrow] (annotation) -- (analyse);
    \end{tikzpicture}

    \vspace{1.5em}

    \begin{tikzpicture}
        \node[fill=white, rounded corners=6pt, inner sep=15pt, drop shadow={opacity=0.08}] {
            \begin{minipage}{0.85\textwidth}
                \begin{tabular}{ll}
                    \faCode\ \textbf{Transcription} & \muted{\small\faGithub\ \href{https://github.com/antoinelemor/Transcribe-tool}{antoinelemor/Transcribe-tool}} \\[0.8em]
                    \faRobot\ \textbf{Annotation LLM} & \muted{\small\faGithub\ \href{https://github.com/antoinelemor/LLM_Tool}{antoinelemor/LLM\_Tool}} \\[0.8em]
                    \faPython\ \textbf{Analyse} & \muted{\small\faGithub\ \href{https://github.com/antoinelemor/nlp-pol}{antoinelemor/nlp-pol}}
                \end{tabular}
            \end{minipage}
        };
    \end{tikzpicture}
\end{frame}

% =============================================================================
% PART 1: RHETORICAL POSTURE
% =============================================================================
\section{Quelle posture rhétorique adoptée ?}

% -----------------------------------------------------------------------------
{
\setbeamertemplate{navigation symbols}{}
\begin{frame}[plain]
    \begin{tikzpicture}[remember picture,overlay]
        \node[anchor=center] at (current page.center) {
            \includegraphics[width=0.95\paperwidth,height=0.92\paperheight,keepaspectratio]{fig1_posture_index_fr.png}
        };
    \end{tikzpicture}
\end{frame}
}

% -----------------------------------------------------------------------------
{
\begin{frame}[plain]
    \begin{tikzpicture}[remember picture,overlay]
        \node[anchor=center] at (current page.center) {
            \includegraphics[width=0.95\paperwidth,height=0.92\paperheight,keepaspectratio]{fig2_posture_timeline_fr.png}
        };
    \end{tikzpicture}
\end{frame}
}

% -----------------------------------------------------------------------------
\begin{frame}{Constat : ce n'est pas Trump le plus agressif}
    \vspace{-1.7em}

    \begin{columns}[T]
        \begin{column}{0.52\textwidth}
            \begin{tikzpicture}
                \node[fill=white, rounded corners=6pt, inner sep=15pt, drop shadow={opacity=0.08}] {
                    \begin{minipage}{0.9\textwidth}
                        Contrairement à l'image habituelle, \textbf{Trump} adopte un ton \emphdata{modéré}.\\[1em]

                        Les plus agressifs ?\\[0.5em]
                        \begin{itemize}
                            \item[\faExclamationTriangle] \emphwarn{Pete Hegseth} \muted{(-0.64)}
                            \item[\faExclamationTriangle] \emphwarn{Marco Rubio} \muted{(-0.49)}
                        \end{itemize}

                        \vspace{1em}
                        \muted{\small Le chef d'état-major Caine reste neutre, il décrit et ne justifie pas.}
                    \end{minipage}
                };
            \end{tikzpicture}
        \end{column}
        \begin{column}{0.45\textwidth}
            \centering
            \vspace{1.5em}

            \begin{tikzpicture}[scale=1.1]
                % Axis
                \draw[muted, thick] (0,0) -- (4.5,0);
                \draw[muted] (2.25,-0.12) -- (2.25,0.12);

                % Labels
                \node[below, font=\tiny] at (0,0) {-2};
                \node[below, font=\tiny] at (2.25,0) {0};
                \node[below, font=\tiny] at (4.5,0) {+1.5};
                \node[above, font=\tiny, danger] at (0.3,0.2) {agressif};
                \node[above, font=\tiny, success] at (4.2,0.2) {pacifique};

                % Gradient background
                \fill[left color=danger!20, right color=success!20, opacity=0.3]
                    (0,-0.08) rectangle (4.5,0.08);

                % Markers with labels
                \fill[green!60!black] (1.07,0) circle (5pt);
                \node[above=8pt, font=\scriptsize] at (1.07,0) {Hegseth};

                \fill[blue!70!black] (1.29,0) circle (5pt);
                \node[below=8pt, font=\scriptsize] at (1.29,0) {Rubio};

                \fill[red!80!black] (1.69,0) circle (5pt);
                \node[above=8pt, font=\scriptsize] at (1.69,0) {Trump};

                \fill[purple!70!black] (2.60,0) circle (5pt);
                \node[below=8pt, font=\scriptsize] at (2.60,0) {Caine};
            \end{tikzpicture}

            \vspace{2em}

            \begin{tikzpicture}
                \node[fill=white, rounded corners=4pt, inner sep=10pt] {
                    \footnotesize
                    \begin{tabular}{lr}
                        \textcolor{green!60!black}{\faCircle} Hegseth & \textbf{-0.64} \\
                        \textcolor{blue!70!black}{\faCircle} Rubio & \textbf{-0.49} \\
                        \textcolor{red!80!black}{\faCircle} Trump & \textbf{-0.27} \\
                        \textcolor{purple!70!black}{\faCircle} Caine & \textbf{+0.15} \\
                    \end{tabular}
                };
            \end{tikzpicture}
        \end{column}
    \end{columns}
\end{frame}

\section{Quels thèmes évoqués par les acteurs ?}

% -----------------------------------------------------------------------------

% -----------------------------------------------------------------------------
{
\begin{frame}[plain]
    \begin{tikzpicture}[remember picture,overlay]
        \node[anchor=center] at (current page.center) {
            \includegraphics[width=0.95\paperwidth,height=0.92\paperheight,keepaspectratio]{fig6_topics_speakers_fr.png}
        };
    \end{tikzpicture}
\end{frame}
}

% -----------------------------------------------------------------------------
\begin{frame}{Constat : une répartition des rôles très claire}
    \vspace{-1.9em}

    \begin{columns}[T]
        \begin{column}{0.48\textwidth}
            \begin{tikzpicture}
                \node[fill=white, rounded corners=6pt, inner sep=12pt, drop shadow={opacity=0.08}] {
                    \begin{minipage}{0.9\textwidth}
                        {\large\textcolor{red!80!black}{\faSquare}\ \textbf{Trump}} \muted{\small n=489}\\[0.5em]
                        Menace sécuritaire \emphdata{25\%}\\
                        Opération militaire 19\%\\
                        Économie/Pétrole 15\%\\
                        \muted{Humanitaire 7\%}
                    \end{minipage}
                };
            \end{tikzpicture}

            \vspace{0.8em}

            \begin{tikzpicture}
                \node[fill=white, rounded corners=6pt, inner sep=12pt, drop shadow={opacity=0.08}] {
                    \begin{minipage}{0.9\textwidth}
                        {\large\textcolor{green!60!black}{\faSquare}\ \textbf{Hegseth}} \muted{\small n=28}\\[0.5em]
                        Opération militaire \emphwarn{57\%}\\
                        Menace sécuritaire 36\%\\
                        \emphwarn{Humanitaire 0\%}
                    \end{minipage}
                };
            \end{tikzpicture}
        \end{column}
        \begin{column}{0.48\textwidth}
            \begin{tikzpicture}
                \node[fill=white, rounded corners=6pt, inner sep=12pt, drop shadow={opacity=0.08}] {
                    \begin{minipage}{0.9\textwidth}
                        {\large\textcolor{blue!70!black}{\faSquare}\ \textbf{Rubio}} \muted{\small n=48}\\[0.5em]
                        Diplomatie \emphdata{40\%}\\
                        Légal/Justice 21\%\\
                        Menace sécuritaire 17\%\\
                        \emphwarn{Humanitaire 0\%}
                    \end{minipage}
                };
            \end{tikzpicture}

            \vspace{0.8em}

            \begin{tikzpicture}
                \node[fill=white, rounded corners=6pt, inner sep=12pt, drop shadow={opacity=0.08}] {
                    \begin{minipage}{0.9\textwidth}
                        {\large\textcolor{purple!70!black}{\faSquare}\ \textbf{Caine}} \muted{\small n=67}\\[0.5em]
                        Opération militaire \emphdata{96\%}\\
                        \muted{Technique, factuel}
                    \end{minipage}
                };
            \end{tikzpicture}
        \end{column}
    \end{columns}

    \vspace{0.8em}

    \centering
    \begin{tikzpicture}
        \node[fill=accent!10, rounded corners=4pt, inner sep=8pt] {
            \small\textcolor{primary}{Seul Trump mentionne l'humanitaire. Hegseth et Rubio : \textbf{jamais}.}
        };
    \end{tikzpicture}
\end{frame}

% -----------------------------------------------------------------------------
\begin{frame}{Interprétation : une justification impérialiste}
    \vspace{-1.7em}
    \vfill
    \centering
    \begin{tikzpicture}
        \node[fill=white, rounded corners=8pt, inner sep=25pt, drop shadow={opacity=0.1}] {
            \begin{minipage}{0.85\textwidth}
                \begin{itemize}
                    \item Les conseillers se concentrent sur la menace et la force
                    \vspace{0.8em}
                    \item Trump seul évoque l'humanitaire : un rôle de façade ?
                    \vspace{0.8em}
                    \item Absence totale de préoccupation humanitaire chez Rubio et Hegseth
                    \item Justification de l'opération majoritairement par la sécurité et l'économie
                \end{itemize}
            \end{minipage}
        };
    \end{tikzpicture}
    \vfill
\end{frame}

\section{Les USA contre le reste du monde?}

% -----------------------------------------------------------------------------


% -----------------------------------------------------------------------------
{
\begin{frame}[plain]
    \begin{tikzpicture}[remember picture,overlay]
        \node[anchor=center] at (current page.center) {
            \includegraphics[width=0.95\paperwidth,height=0.92\paperheight,keepaspectratio]{fig7_us_vs_them_fr.png}
        };
    \end{tikzpicture}
\end{frame}
}

% -----------------------------------------------------------------------------
\begin{frame}{Constat : un discours fortement manichéen}
    \vspace{-1.5em}

    \begin{columns}[T]
        \begin{column}{0.48\textwidth}
            \begin{tikzpicture}
                \node[fill=success!8, rounded corners=8pt, inner sep=15pt] {
                    \begin{minipage}{0.85\textwidth}
                        \centering
                        {\large\bfseries ``Nous''}\\
                        \muted{\small États-Unis}\\[1em]
                        {\Huge\textcolor{success}{\textbf{+0.39}}}\\[0.8em]
                        \begin{tabular}{lr}
                            Positif & \textbf{48\%} \\
                            Neutre & 42\% \\
                            Négatif & 9\% \\
                        \end{tabular}
                    \end{minipage}
                };
            \end{tikzpicture}
        \end{column}
        \begin{column}{0.48\textwidth}
            \begin{tikzpicture}
                \node[fill=danger!8, rounded corners=8pt, inner sep=15pt] {
                    \begin{minipage}{0.85\textwidth}
                        \centering
                        {\large\bfseries ``Eux''}\\
                        \muted{\small Entités étrangères}\\[1em]
                        {\Huge\textcolor{danger}{\textbf{-0.53}}}\\[0.8em]
                        \begin{tabular}{lr}
                            Positif & 2\% \\
                            Neutre & 43\% \\
                            Négatif & \textbf{55\%} \\
                        \end{tabular}
                    \end{minipage}
                };
            \end{tikzpicture}
        \end{column}
    \end{columns}

    \vspace{0.2em}

    \centering
    \begin{tikzpicture}
        \node[fill=white, rounded corners=6pt, inner sep=12pt, drop shadow={opacity=0.1}] {
            {\large Indice d'animosité : \emphwarn{-0.46}} \quad \muted{(élevé)}
        };
    \end{tikzpicture}
\end{frame}

% -----------------------------------------------------------------------------
\begin{frame}{Interprétation : un isolationnisme impéraliste ?}
    \vspace{-1.7em}
    \vfill
    \centering
    \begin{tikzpicture}
        \node[fill=white, rounded corners=8pt, inner sep=25pt, drop shadow={opacity=0.1}] {
            \begin{minipage}{0.85\textwidth}
                \begin{itemize}
                    \item Vision binaire du monde : ``nous'' (positif) vs ``eux'' (négatif)
                    \vspace{0.8em}
                    \item Un interventionnisme régional assumé
                    \vspace{0.8em}
                    \item Un discours binaire qui sert à justifier des actions impérialistes
                \end{itemize}
            \end{minipage}
        };
    \end{tikzpicture}
    \vfill
\end{frame}

% =============================================================================
% PART 2: Q&A ANALYSIS
% =============================================================================
\section{Face aux journalistes}

% -----------------------------------------------------------------------------
{
\begin{frame}[plain]
    \begin{tikzpicture}[remember picture,overlay]
        \node[anchor=center] at (current page.center) {
            \includegraphics[width=0.95\paperwidth,height=0.92\paperheight,keepaspectratio]{fig3_responses_fr.png}
        };
    \end{tikzpicture}
\end{frame}
}

% -----------------------------------------------------------------------------
\begin{frame}{Constat : des réponses rarement directes}
    \vspace{-1em}

    \begin{columns}[c]
        \begin{column}{0.45\textwidth}
            \centering
            \begin{tikzpicture}
                \node[fill=danger!10, rounded corners=12pt, inner sep=25pt] {
                    \begin{minipage}{0.7\textwidth}
                        \centering
                        {\fontsize{48}{52}\selectfont\textcolor{danger}{\textbf{72\%}}}\\[0.5em]
                        {\large réponses \textbf{partielles}}
                    \end{minipage}
                };
            \end{tikzpicture}

            \vspace{1.5em}

            \begin{tikzpicture}
                \node[fill=white, rounded corners=6pt, inner sep=12pt] {
                    \begin{tabular}{rl}
                        \emphok{21\%} & réponses directes \\[0.3em]
                        7\% & changements de sujet \\
                    \end{tabular}
                };
            \end{tikzpicture}
        \end{column}
        \begin{column}{0.5\textwidth}
            \begin{tikzpicture}
                \node[fill=white, rounded corners=6pt, inner sep=18pt, drop shadow={opacity=0.08}] {
                    \begin{minipage}{0.9\textwidth}
                        {\large Les journalistes obtiennent des \textbf{fragments}, rarement des réponses complètes.}\\[1.5em]
                        \muted{Sur 57 segments de réponse, seuls 12 répondent directement à la question posée.}
                    \end{minipage}
                };
            \end{tikzpicture}
        \end{column}
    \end{columns}
\end{frame}

% -----------------------------------------------------------------------------
{
\begin{frame}[plain]
    \begin{tikzpicture}[remember picture,overlay]
        \node[anchor=center] at (current page.center) {
            \includegraphics[width=0.95\paperwidth,height=0.92\paperheight,keepaspectratio]{fig5_progressive_fr.png}
        };
    \end{tikzpicture}
\end{frame}
}

% -----------------------------------------------------------------------------
\begin{frame}{Constat : le pétrole avant tout}
    \vspace{-0.5em}

    \centering
    {\large Un décalage thématique révélateur}

    \vspace{0.8em}

    \begin{columns}[T]
        \begin{column}{0.48\textwidth}
            \begin{tikzpicture}
                \node[fill=accent!8, rounded corners=8pt, inner sep=15pt] {
                    \begin{minipage}{0.85\textwidth}
                        {\large\bfseries\faNewspaper\ Les journalistes}\\[0.8em]
                        \begin{enumerate}
                            \item Diplomatie
                            \item Gouvernance
                            \item Légal / Justice
                        \end{enumerate}
                        \vspace{0.8em}
                        \muted{\small Questions sur le \textit{comment}}
                    \end{minipage}
                };
            \end{tikzpicture}
        \end{column}
        \begin{column}{0.48\textwidth}
            \begin{tikzpicture}
                \node[fill=danger!8, rounded corners=8pt, inner sep=15pt] {
                    \begin{minipage}{0.85\textwidth}
                        {\large\bfseries\faUser\ Trump répond}\\[0.8em]
                        \begin{enumerate}
                            \item \emphwarn{Économie / Pétrole}
                            \item Opération militaire
                            \item Diplomatie
                        \end{enumerate}
                        \vspace{0.8em}
                        \muted{\small Insistance sur les \textit{ressources}}
                    \end{minipage}
                };
            \end{tikzpicture}
        \end{column}
    \end{columns}

    \vspace{1em}

    \centering
    \begin{tikzpicture}
        \node[fill=white, rounded corners=6pt, inner sep=12pt, drop shadow={opacity=0.1}] {
            \large Trump préoccupé par les \textbf{ressources} avant tout
        };
    \end{tikzpicture}
\end{frame}

% -----------------------------------------------------------------------------
{
\begin{frame}[plain]
    \begin{tikzpicture}[remember picture,overlay]
        \node[anchor=center] at (current page.center) {
            \includegraphics[width=0.95\paperwidth,height=0.92\paperheight,keepaspectratio]{fig4_evasion_fr.png}
        };
    \end{tikzpicture}
\end{frame}
}

% -----------------------------------------------------------------------------
\begin{frame}{Constat : le ``comment'' reste sans réponse}
    \vspace{-1em}

    \centering
    {\large Les sujets les plus \textbf{évités} sont les plus \textbf{concrets}}

    \vspace{1em}

    \begin{columns}[T]
        \begin{column}{0.48\textwidth}
            \begin{tikzpicture}
                \node[fill=danger!8, rounded corners=6pt, inner sep=12pt] {
                    \begin{minipage}{0.85\textwidth}
                        \centering
                        {\Large\emphwarn{80\%}}\\[0.2em]
                        \textbf{Gouvernance}\\
                        \muted{\footnotesize réponse partielle}
                    \end{minipage}
                };
            \end{tikzpicture}

            \vspace{0.6em}

            \begin{tikzpicture}
                \node[fill=danger!7, rounded corners=6pt, inner sep=12pt] {
                    \begin{minipage}{0.85\textwidth}
                        \centering
                        {\Large\emphwarn{77\%}}\\[0.2em]
                        \textbf{Diplomatie}\\
                        \muted{\footnotesize réponse partielle}
                    \end{minipage}
                };
            \end{tikzpicture}
        \end{column}
        \begin{column}{0.48\textwidth}
            \begin{tikzpicture}
                \node[fill=danger!6, rounded corners=6pt, inner sep=12pt] {
                    \begin{minipage}{0.85\textwidth}
                        \centering
                        {\Large\emphwarn{67\%}}\\[0.2em]
                        \textbf{Légal / Justice}\\
                        \muted{\footnotesize réponse partielle}
                    \end{minipage}
                };
            \end{tikzpicture}

            \vspace{0.6em}

            \begin{tikzpicture}
                \node[fill=subtle, rounded corners=6pt, inner sep=12pt] {
                    \begin{minipage}{0.85\textwidth}
                        \centering
                        {\Large\emphwarn{33\%}}\\[0.2em]
                        \textbf{Économie / Pétrole}\\
                        \muted{\footnotesize changement de sujet/appel à conseiller}
                    \end{minipage}
                };
            \end{tikzpicture}
        \end{column}
    \end{columns}

    \vspace{1em}

    \muted{Seule exception : politique intérieure US (\emphok{92\%} de réponses directes)}
\end{frame}

% =============================================================================
% CONCLUSION
% =============================================================================
\section{Conclusion}

% -----------------------------------------------------------------------------
\begin{frame}{Ce que révèle l'analyse}
    \vspace{-1.3em}

    \begin{columns}[T]
        \begin{column}{0.48\textwidth}
            \begin{tikzpicture}
                \node[fill=white, rounded corners=8pt, inner sep=14pt, drop shadow={opacity=0.1}] {
                    \begin{minipage}{0.9\textwidth}
                        {\large\textcolor{accent}{\faUsers}\ \textbf{Agressivité déléguée}}\\[0.4em]
                        Trump modéré, conseillers offensifs. Le président dirige, poussé par son entourage.
                    \end{minipage}
                };
            \end{tikzpicture}

            \vspace{0.6em}

            \begin{tikzpicture}
                \node[fill=white, rounded corners=8pt, inner sep=14pt, drop shadow={opacity=0.1}] {
                    \begin{minipage}{0.9\textwidth}
                        {\large\textcolor{accent}{\faGlobe}\ \textbf{Impérialisme clair}}\\[0.4em]
                        Interventionnisme régional assumé. Vision binaire ``nous'' vs ``eux''.
                    \end{minipage}
                };
            \end{tikzpicture}
        \end{column}
        \begin{column}{0.48\textwidth}
            \begin{tikzpicture}
                \node[fill=white, rounded corners=8pt, inner sep=14pt, drop shadow={opacity=0.1}] {
                    \begin{minipage}{0.9\textwidth}
                        {\large\textcolor{accent}{\faOilCan}\ \textbf{Le pétrole avant tout}}\\[0.4em]
                        Décalage thématique révélateur. Justification par les ressources, pas l'humanitaire.
                    \end{minipage}
                };
            \end{tikzpicture}

            \vspace{0.6em}

            \begin{tikzpicture}
                \node[fill=white, rounded corners=8pt, inner sep=14pt, drop shadow={opacity=0.1}] {
                    \begin{minipage}{0.9\textwidth}
                        {\large\textcolor{accent}{\faQuestionCircle}\ \textbf{L'avenir reste flou}}\\[0.4em]
                        Gouvernance, légalité : aucun plan concret pour l'après.
                    \end{minipage}
                };
            \end{tikzpicture}
        \end{column}
    \end{columns}
\end{frame}

% -----------------------------------------------------------------------------
\begin{frame}[plain]
    \vfill
    \begin{center}
        \begin{tikzpicture}
            \node[fill=white, rounded corners=12pt, inner sep=30pt, drop shadow={opacity=0.15}] {
                \begin{minipage}{0.6\textwidth}
                    \centering
                    {\bfseries Antoine Lemor}\\[0.5em]
                    \muted{\small\faGithub\ \href{https://github.com/antoinelemor}{github.com/antoinelemor}}\\[1.5em]
                    \textcolor{subtle}{\rule{0.6\textwidth}{0.5pt}}\\[1.2em]
                    {\small Vous voulez voir la méthode et le code ?}\\[1em]
                    \begin{tabular}{@{}rl@{}}
                    \faChartBar\ \textbf{Cette analyse} & \muted{\small\href{https://github.com/antoinelemor/nlp-pol}{antoinelemor/nlp-pol}} \\[0.5em]
                    \faCode\ \textbf{Transcription} & \muted{\small\href{https://github.com/antoinelemor/Transcribe-tool}{antoinelemor/Transcribe-tool}} \\[0.5em]
                    \faRobot\ \textbf{Annotation LLM} & \muted{\small\href{https://github.com/antoinelemor/LLM_Tool}{antoinelemor/LLM\_Tool}}
                    \end{tabular}
                \end{minipage}
            };
        \end{tikzpicture}
    \end{center}
    \vfill
\end{frame}

\end{document}
